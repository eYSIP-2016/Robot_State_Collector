\documentclass[a4paper,12pt,oneside]{book}

%-------------------------------Start of the Preable------------------------------------------------
\usepackage[english]{babel}
\usepackage{blindtext}
%packagr for hyperlinks
\usepackage{hyperref}
\hypersetup{
    colorlinks=true,
    linkcolor=blue,
    filecolor=magenta,      
    urlcolor=cyan,
}

\urlstyle{same}
%use of package fancy header
\usepackage{fancyhdr}
\setlength\headheight{26pt}
\fancyhf{}
%\rhead{\includegraphics[width=1cm]{logo}}
\lhead{\rightmark}
\rhead{\includegraphics[width=1cm]{logo}}
\fancyfoot[RE, RO]{\thepage}
\fancyfoot[CE, CO]{\href{http://www.e-yantra.org}{www.e-yantra.org}}

\pagestyle{fancy}

%use of package for section title formatting
\usepackage{titlesec}
\titleformat{\chapter}
  {\Large\bfseries} % format
  {}                % label
  {0pt}             % sep
  {\huge}           % before-code
 
%use of package tcolorbox for colorful textbox
\usepackage[most]{tcolorbox}
\tcbset{colback=cyan!5!white,colframe=cyan!75!black,halign title = flush center}

\newtcolorbox{mybox}[1]{colback=cyan!5!white,
colframe=cyan!75!black,fonttitle=\bfseries,
title=\textbf{\Large{#1}}}

%use of package marginnote for notes in margin
\usepackage{marginnote}

%use of packgage watermark for pages
%\usepackage{draftwatermark}
%\SetWatermarkText{\includegraphics{logo}}
\usepackage[scale=2,opacity=0.1,angle=0]{background}
\backgroundsetup{
contents={\includegraphics{logo}}
}

%use of newcommand for keywords color
\usepackage{xcolor}
\newcommand{\keyword}[1]{\textcolor{red}{\textbf{#1}}}

%package for inserting pictures
\usepackage{graphicx}

%package for highlighting
\usepackage{color,soul}

%new command for table
\newcommand{\head}[1]{\textnormal{\textbf{#1}}}


%----------------------End of the Preamble---------------------------------------


\begin{document}

%---------------------Title Page------------------------------------------------
\begin{titlepage}
\raggedright
{\Large eYSIP2016\\[1cm]}
{\Huge\scshape Robot State Collector \\[.1in]}
\vspace*{\fill}
\begingroup
\centering

\Large \textbf{PROTOCOL}

\endgroup
\vspace*{\fill}
\begin{flushright}
{\large Amanpreet Singh \\}
{\large Amit Raushan \\}
{\large Shubham Gupta \\}
{\large Duration of Internship: $ 21/05/2016-10/07/2016 $ \\}
\end{flushright}

{\itshape 2016, e-Yantra Publication}
\end{titlepage}
%-------------------------------------------------------------------------------

\chapter[Project Tag]{List of all the Protocols}

\begin{enumerate}
    \item The header files provided must be in the same folder as that of the user's code for effectivaely merging the state collection code.
    \item In the GUI provided to the user there has to be a "ip.txt" file which contains the static IP of the server and a "publicKey.txt" which contains the Public Key which would be used to encrypt the Symmetric Key before sending it to the server.  
    \item In Client Side Code changes have to be made to use the GUI for reading information at a different Baud Rate. \\ \\ 
    Inside the function SimpleRead(String com) there is a line as follows:\\
    serialPort.setSerialPortParams(9600,SerialPort.DATABITS\_8,SerialPort.STOPBITS\_1, SerialPort.PARITY\_NONE);\\
    The required Baud Rate can be written instead of 9600 to read data coming with that Baud Rate.
    \item To change the number of readings being sent to the GUI following changes need to be made. \\ \\
    Inside a function called run() there is a command as follows:\\
    if (count == 48) \\ 
    Just change 48 to (4*number of readings per period) to prevent the haphazard storing of data.
    \item To change the time after which state is being collected, configure the timer in state collection header file accordingly.
\end{enumerate}


\end{document}

